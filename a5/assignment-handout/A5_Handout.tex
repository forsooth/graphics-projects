\documentclass[10pt,twocolumn]{article}
\usepackage[landscape,margin=0.75in]{geometry}                % See geometry.pdf to learn the layout options. There are lots.
\geometry{letterpaper}                   % ... or a4paper or a5paper or ... 
%\geometry{landscape}                % Activate for for rotated page geometry
%\usepackage[parfill]{parskip}    % Activate to begin paragraphs with an empty line rather than an indent
\setlength{\columnsep}{1cm}
\usepackage{graphicx}
\usepackage{amssymb}
\usepackage{epstopdf}
\usepackage[usenames]{color}
\usepackage{titlesec}
\usepackage{hyperref}

\definecolor{light-gray}{gray}{0.45}
\titleformat{\section}
{\color{black}\normalfont\huge\bfseries}
{\color{black}\thesection}{1em}{}

\titleformat{\subsection}
{\color{light-gray}\normalfont\Large\bfseries}
{\color{light-gray}\thesubsection}{1em}{}

\DeclareGraphicsRule{.tif}{png}{.png}{`convert #1 `dirname #1`/`basename #1 .tif`.png}

\title{\Huge{\bf Assignment 5: Ray}}
\author{Comp175: Introduction to Computer Graphics -- Spring 2016}
\date{Algorithm due:  {\bf Monday April 11th} at 11:59pm\\
Project due:  {\bf Monday April 18th} at 11:59pm}                                           % Activate to display a given date or no date

\begin{document}
\maketitle
%\section{}
%\subsection{}

\section{Introduction}								
In Intersect, you saw a glimpse of what you could do with a rendering algorithm that stressed quality over speed. Truly curved surfaces are possible, and everything has a sort of ``perfect'' feel to it. As you may have noticed, there are a few things that our renderer does not yet handle. For example, what happens if you have a shiny surface? What about texture mapping? Shadows? These issues and more are addressed in this assignment. In {\tt Ray}, you will be filling out the renderer you wrote for {\tt Intersect} to support reflections, texture mapping, shadows, more advanced lighting, and perhaps even transparency, motion blur, spacial subdivision, or bump mapping.\\\\The demo for this project is similar to {\tt Intersect}. Try some of the new XML scene files, and try to implement some of the more advanced options in ray tracing.\\\\There is no additional support code for {\tt Ray}; in this assignment, you are building entirely on your {\tt Intersect} code.

\section{Requirements}
For this project you are required to extend your {\tt Intersect} code to implement additional features in your ray tracer. Your ray tracer must be able to handle:
\begin{itemize}
\item Reflection
\item Texture mapping for, at least, the cube, cylinder, cone, and sphere 
\item Specular highlights
\item Shadows
\item Point vs. directional lighting
%\item Light fall-off (attenuation)
\end{itemize}
To calculate the intensity of the reflection value, you need to determine the reflection vector based on an object's normal and the look vector. You then need to recursively calculate the intensity at the intersection point where the reflection vector hits. With each successive recursive iteration, the contribution of the reflection to the overall intensity drops off. For this reason, you need to set a limit for the amount of recursion with which you calculate your reflection. You must make it possible to change the recursion limit easily\footnote{By ``easily,'' we mean that you should have this as an interface option.} because I may want to change it during grading. Also, you will want to terminate the recursion when the intensity of the contributed color drops below a reasonable threshold.\\\\To review, the lighting model you will be implementing is:

\[I_{\lambda} = k_aO_{a\lambda} + \sum_{i=1}^{m} \left[f_{att_i}I_{i\lambda}*(k_dO_{d\lambda}(\hat{N}\cdot\hat{L}_i) + k_sO_{s\lambda}(\hat{R}_i\cdot\hat{V})^f\right] + k_sO_{r\lambda}I_{r\lambda}\]
Here, the subscripts $a$, $d$, $s$, and $r$ stand for ambient, diffuse, specular, and reflected, respectively. \\\\
\begin{tabular}{rcp{3.3in}}
$I_{\lambda}$ & = & final intensity for wavelength $\lambda$; in our case the final R, G, or B value of the pixel we want to color\\\\
$k$ & = & a constant coefficient. For example, $k_a$ is the global intensity of ambient light; {\small{\tt SceneGlobalData::ka}} in the support code\\\\
$O$ &=& the object being hit my the ray. For example, $O_{d\lambda}$ is the diffuse color at the point of ray intersection on the object\\\\
$m$ &=& the number of lights in the scene\\\\
$f_{att_i}$ &=& the attenuation for light $i$ (which you are not required to do for this assignment) \\\\
$I_{i\lambda}$ &=& intensity of light $i$ for wavelength $\lambda$\\\\
$\hat{N}$ &=& the unit length surface normal at the point of intersection\\\\
$\hat{L}_i$ &=& the unit length incoming light vector from light $i$\\\\
$\hat{R}_i$ &=& the unit length reflected light from light $i$\\\\
$\hat{V}$ &=& the normalized line of sight\\\\
$f$ &=& the specular component\\\\
$I_{r\lambda}$ & = & the intensity of the reflected light
\end{tabular}

\section{Testing}
Your ray tracer's output should look like the demo (for a given scene file and render settings). In addition,  there are new scenefiles on the course website that specifically test the new features in {\tt Ray}. That said, you should not ignore the other scenefiles. Make sure all of the other files are also rendered properly. Also, pay attention to the orientation of your textures.
\\\\You should certainly try to render the scene that you created for SceneView, but feel free to add on top of it now that you have more control over lighting, shading, and colors.%\\\\If you create your own additional scene files, please submit them along with your code. We'll be awarding extra credit for particularly interesting scenes.
\section{FAQ}
\subsection{Texture Mapping SNAFUs}
When texture mapping planes you need to be careful. If you're texture mapping the negative z face of the cube, you'll be mapping the intersection point's x position to the $u$ in $(u,v)$ space. The problem is when you go left-to-right on that face, your $x$ values are actually going from positive to negative. This isn't the only cube face that something like this will happen on, so check each face to make sure.\\\\To texture map the cone, just do it the same way as a cylinder (except there's only one cap, of course).

\section{Extra Credit}
{\tt Ray} is one of the coolest projects you'll ever write at Tufts. You, yes, you, can make it even cooler by doing some sweet extra credit. Here are some ideas (book sections are included if there's significant discussion of the topic).
\begin{itemize}
\item{\bf Antialiasing} Brute force supersampling isn't hard to do and antialiased images look really sexy. If you're feeling brave, try your hand at adaptive supersampling or stochastic supersampling
\item{\bf Transparency and Refraction}
\item{\bf Motion blur}
\item{\bf Depth-of-field}
\item{\bf Fewer intersection tests} Bounding volumes, hierarchical bounding volumes, octrees, and kd-trees are all things to try that will get big speedups on complex scenes since most of the clock cycles go to intersection tests. Mucho bonus points if you do one of these. Really fast intersection tests might get you a few points too, but only if they're really good. It is highly encouraged that you do some sort of spacial subdivision, but certainly not required at all.
\item{\bf Constructive solid geometry} (a.k.a. CSG)
\item{\bf Bump mapping} Like texture mapping, except each texel contains information about the normals instead of color values. It's a great way to add geometry to an object without having to actually render the geometry.
\item{\bf Texture mapping and/or intersecting other shapes}, like the torus
%\item{\bf Spotlights} Spotlights have position, direction, and an aperture in degrees. If a light is a spotlight, {\tt SceneLightData}'s {\tt m\_type} will be equal to {\tt LIGHT\_SPOT} and {\tt m\_aperture} will contain the aperture size.
\item{\bf Optimizations} Be careful here. ``Premature optimization is the root of all evil''\footnote{Knuth}. You'll learn that he's right at some point in your career, but let's not learn that lesson on {\tt Ray}. Get the basic functionality done then go for the gusto. This isn't as important as it once was now that we have blazing fast machines, admittedly, but it's still awful fun. Multithreading Raytracing is embarrassingly parallel because a ray does not depend on the outcome of any of the other rays. Each ray cast per scanline can be made into its own thread.
\item{\bf Texture filtering} (bilinear, trilinear, what have you)
\item Whatever else you can think of!
\end{itemize}
Remember to make a few scenefiles to show off the extra credit you do. If you create your own additional scene files, please provide them so that other students may benefit from them. We'll also be awarding extra credit for particularly interesting scenes.

\section{How to Submit}					
Complete the algorithm portion of this assignment with your teammate. You may use a calculator or computer algebra system. All your answers should be given in simplest form. When a numerical answer is required, provide a reduced fraction (i.e. 1/3) or at least three decimal places (i.e. 0.333).  Show all work.

For the project portion of this assignment, you are encouraged to discuss your strategies with your classmates. However, all team must turn in ORIGINAL WORK, and any collaboration or references outside of your team must be cited appropriately in the header of your submission.

Hand in the assignment using the following commands:
\begin{itemize}
\item Algorithm:  \ \ \ \ \ \ {\tt provide comp175 a5-alg}
\item Project code: \ \ \ {\tt provide comp175 a5}\\
\end{itemize}

\end{document}  
