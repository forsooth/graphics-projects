\documentclass[10pt,twocolumn]{article}
\usepackage[margin=0.75in]{geometry}                % See geometry.pdf to learn the layout options. There are lots.
\geometry{letterpaper}                   % ... or a4paper or a5paper or ... 
%\geometry{landscape}                % Activate for for rotated page geometry
%\usepackage[parfill]{parskip}    % Activate to begin paragraphs with an empty line rather than an indent
\setlength{\columnsep}{1cm}
\usepackage{graphicx}
\usepackage{amssymb}
\usepackage{epstopdf}
\usepackage{fullpage}
\usepackage[usenames]{color}
\usepackage{titlesec}
\usepackage{hyperref}
\usepackage{framed}

\definecolor{light-gray}{gray}{0.45}
\titleformat{\section}
{\color{black}\normalfont\huge\bfseries}
{\color{black}\thesection}{1em}{}

\titleformat{\subsection}
{\color{light-gray}\normalfont\Large\bfseries}
{\color{light-gray}\thesubsection}{1em}{}

\DeclareGraphicsRule{.tif}{png}{.png}{`convert #1 `dirname #1`/`basename #1 .tif`.png}

\title{\Huge{\bf Algorithm 1: Shapes}}
\author{Comp175: Computer Graphics -- Spring 2016}
\date{Due:  {\bf Monday February 8} at 11:59pm}                                           % Activate to display a given date or no date

\begin{document}
\maketitle
%\section{}
%\subsection{}

\begin{verbatim}
Your Names: __Matt_Asnes_________________

            __Fury_Sheron________________

Your CS Logins: __masnes01_______________

                __lshero01_______________\end{verbatim}

\section{Instructions}
Complete this assignment only with your teammate. You may use a
calculator or computer algebra system. All your answers should be given in simplest form.
When a numerical answer is required, provide a reduced fraction (i.e. 1/3) or at least three
decimal places (i.e. 0.333). Show all work; write your answers on this sheet. This algorithm handout is worth 2\% of your final grade for the class.


\section{Cube}
 {\bf [1 point]} Take a look at one face of the cube. Change the tessellation parameter. How do the number of small squares against one edge correspond to the tessellation parameter?
\begin{framed}
They correspond in a 1-to-1 ratio. i.e if the tesselation parameter was 4, there would be 4 small squares against an edge.
\end{framed}
{\bf [1 point]} Imagine a unit cube at the origin with tessellation parameter 2. Its front face lies in the +XY plane. What are the normal vectors that correspond with each of the eight triangles that make up this face? (Note: when asked for a normal, you should always give a normalized vector, meaning a vector of length one.)
\begin{framed}
All of them will be $(0, 0, 1)$.
\end{framed}


\section{Cylinder}
{\bf [1.5 points]} The caps of the cylinder are regular polygons with $N$ sides, where $N$'s value is determined by parameter 2 ($p_2$). You will notice they are cut up like a pizza with $N$ slices, which are isosceles triangles. The vertices of the $N$-gon lie on a perfect circle. What is the equation of the circle that they lie on  in terms of the radius (0.5) and the angle $\theta$?
\begin{framed}
In parametric form, the x and y coordinates can be generalized in terms of radius and theta such that:
\[x = r \cdot \cos(\theta)\]
\[y= r \cdot \sin(\theta)\]
\[\textrm{for } r = 0.5\]
\end{framed}

{\bf [1.5 points]} What is the surface normal of an arbitrary point along the barrel of the cylinder? It might be easier to think of this problem in cylindrical coordinates, and then transform your answer to cartesian after you have solved it in cylindrical coords.
\begin{framed}
There will be no $z$-component in the normal for any point on the barrel because any face of the barrel will be parallel to the $z$-axis.
As such, any surface normal on the barrel of the cylindar will point in the direction of $\theta$.
Therefore, at any point, the normal $\vec{N}= \Big(\frac{\cos(\theta)}{r}, \frac{\sin(\theta)}{r}, 0\Big)$
\end{framed}


\section{Cone}
{\bf [1 point]} Look at the cone with Y-axis rotation = 0 degrees, and X-axis rotation = 0 degrees. How many triangles make up one of the $segmentX$ ``sides'' of the cone when $segmentY=1$? When $segmentY=2$? 3? $n$?
\begin{framed}
It is a function of $segmentY$:
\[(2segmentY - 1)\]
Such that when $segmentY = 1$, there is 1 triangle, when $segmentY = 2$, there are 3 triangles, and when $segmentY = 3$, there are 5 triangles. This accurately reflects the behavior displayed in the demo.
\end{framed}

{\bf [1 point]} What is the surface normal at the tip of the cone? Keep in mind that a singularity does not have a normal; this implies that there will not be a unique normal at the tip of the cone. You can achieve a good shading effect by thinking of $segmentX$ vectors with their base at the tip of the cone, each pointing outward, normal from the face of the triangle associated with it along the side of the cone. Think about how OpenGL can use this information to make a realistic point at the top of the cone, and draw a simple schematic sketch illustrating the normal for one of the triangles at the tip. (As long as it is clear that you get the idea, you will receive full credit.)
\begin{framed}
\centering
\includegraphics[scale=.5]{conenorms}
There is no unique surface normal at the tip. Rather, every face that converges as part of the tip will have a normal comprised of an $x$-component based off of the theta of the face and a $y$-component that's the normalized vertical component shared by all converging faces. The normal that makes most sense for the tip begins at the tip and points vertically upward; This normal is the average of all the normals aforementioned faces that comprise the cone.
Additionally, any horizontal face constructed by slicing off the tip of the cone would have a normal pointing this direction, so it would follow that even as the size of this face goes to zero, the normal stays the same.
\end{framed}

{\bf [1 point]} Take the two dimensional line formed by the points $(0, 0.5)$ and $(0.5, -0.5)$ and find its slope $m$.
\begin{framed}
By the equation to calculate slope,
\[m = \frac{\Delta x}{\Delta y} = \frac{0 - 0.5}{0.5 - (-0.5)} = -2\]
\end{framed}


{\bf [1 point]} ${{-1}\over{m}}$ is the slope perpendicular to this line. Using this slope, we
can find the vertical and radial/horizontal components of the normal on the cone body. The radial/horizontal component is the component in the XZ plane. What is the {\bf magnitude} of this component in a normalized normal vector?
\begin{framed}
\centering
\includegraphics[scale=.25]{conemath}
We know the slope of the normal will be $\frac{1}{2}$ and its magnitude will be 1. Then the radial/horizontal component (which we have rendered in red), will have a magnitude twice that of the vertical component; since the hypotenuse of the triangle they form has length 1 (since the vector is normalized), if we call the length of the vertical component $x$ then $1^2 = (2x)^2 + x^2$, and therefore $x = \frac{1}{\sqrt{5}}$. Then the horizontal component has magnitude $2x = \frac{2\sqrt{5}}{5}$. 
\end{framed}


{\bf [1 point]} The component in the $y$ direction is the vertical component. What is the {\bf magnitude} of this component in a normalized normal vector?
\begin{framed}
As above, the vertical component has magnitude $\frac{\sqrt{5}}{5}$.
\end{framed}


\section{Sphere}
The sphere in the demo is tessellated in the latitude/longitude manner, so the points you want to calculate are straight spherical coordinates. The two parameters can be used as $\theta$ and $\phi$, or longitude and latitude. Recall, that the conversion from spherical to Cartesian coordinates is given by
\begin{eqnarray*}
x & = & r * \sin{\phi} * \cos{\theta}\\
y & = & r * \cos{\phi}\\
z & = & r * \sin{\phi}*\sin{\theta}
\end{eqnarray*}
{\bf [1 point]} What is the surface normal of the sphere at an arbitrary surface point $(x,y,z)$?
\begin{framed}
We can write the normal vector on a sphere as a function of the point, i.e.
\[N(x, y, z) = \Big(\frac{\sin{\phi} * \cos{\theta}}{r}, \frac{\cos{\phi}}{r}, \frac{\sin{\phi}*\sin{\theta}}{r}\Big)\]
the normal vector at a point points in the same direction as the vector from the center
of the sphere to the point, and we have to make sure to normalize it by dividing each component by $r^2$.
\end{framed}


\section{How to Submit}

Hand in a PDF version of your solutions using the following command:
\begin{center}
 {\tt provide comp175 a1-alg}
 \end{center}
\end{document}  
